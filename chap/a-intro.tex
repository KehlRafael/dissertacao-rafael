\chapter{Introdução}

A teoria dos jogos evolucionária surgiu no início da década de $70$ com o objetivo de explicar a existência de comportamentos ritualizados que animais mostram em situações de conflito. Até então, acreditava-se que esse tipo de comportamento surgiu por ser benéfico à espécie, porém essa ideia vai de encontro ao pensamento Darwinista, onde a seleção ocorre em nível individual \cite{maynardWoS}. Nesse contexto, John Maynard Smith, inspirado por George Price, buscou respostas para esse problema na Teoria dos Jogos e, juntos, publicaram o livro The Logic of Animal Conflict \cite{smith1973logic} que introduziu os Jogos Evolucionários e a definição de estados evolucionariamente estáveis.

A principal diferença entre jogos clássicos e evolucionários é que a versão evolucionária não requer que os jogadores envolvidos sejam racionais, mas somente que eles possuam uma estratégia. Essa estratégia é, então, testada através de um jogo e o pagamento é o direito de reprodução, onde seus descendentes irão herdar sua estratégia. Assim, o sucesso de uma estratégia depende não só de seu desempenho contra as estratégias adversárias, incluindo a própria estratégia, mas também da frequência que essas estratégias são encontradas na população. Na década que sucedeu a publicação do trabalho de Price e Smith houve um crescimento contínuo nas aplicações de jogos evolucionários dentro da biologia e Smith fez uma coletânea de alguns resultados iniciais no livro Evolution and the Theory of Games \cite{smith1982evolution}.

Neste mesmo período, Taylor e Jonker modelaram a dinâmica dos jogos evolucionários no caso contínuo, com um sistema de equações diferenciais não lineares de primeira ordem, e no caso discreto, com um sistema de equações de diferenças não lineares, no artigo Evolutionarily Stable Strategies and Game Dynamics \cite{taylor1978evolutionary}. Essa dinâmica ficou conhecida como dinâmica do replicador e os sistemas de equações como equação de replicação. Essa modelagem permitiu a generalização da definição de estados evolucionariamente estáveis, a discussão da estabilidade das soluções desses sistemas e, posteriormente, a generalização dos equilíbrios de Nash para jogos evolucionários.

A equação de replicação também foi usada para descrever desde fenômenos biológicos, como mutação e espalhamento viral \cite{nowak2006evolutionary}, até dinâmicas socio-econômicas, como cooperação \cite{nowak2006evolutionary} e difusão de conhecimento e riqueza na formação de redes sociais \cite{ehrhardt2006diffusion}. Neste contexto, Madeo e Mocenni introduziram uma generalização da dinâmica do replicador para populações em redes finitas \cite{madeo2015}, possibilitando analisar os efeitos que restrições topológicas têm nos fenômenos e dinâmicas já conhecidos.

No contexto sócio-econômico e com uma perspectiva evolucionária, jogadores tomam decisões que, em média, resultam em aumentos no seu pagamento. Nesse sentido, é dito que jogadores evolucionários exibem uma racionalidade profunda \cite{kenrick2009deep}. Com base nisso, diremos que jogadores em jogos evolucionários clássicos são racionais, pois tomam decisões (mudam de estratégia) para aumentar seu pagamento. Este trabalho tem como objetivo introduzir jogadores hiper-racionais \cite{askari2019behavioral} na dinâmica de replicação para redes finitas. O comportamento humano muitas vezes não se encaixa na hipótese de racionalidade \cite{smith1982evolution, sep-game-evolutionary}, pois humanos possuem sentimentos complexos que interferem em suas tomadas de decisão. Introduziremos a hiper-racionalidade através de um novo parâmetro chamado matriz de preferências, uma matriz cuja entrada $(i,j)$ representa a importância que o jogador $i$ dá para o pagamento recebido por $j$ no cálculo do seu próprio pagamento. Usaremos esse parâmetro para modelar sentimentos como altruísmo, devoção, desconfiança e ciúmes. 

No capítulo 2 desenvolveremos a teoria de equações diferenciais para que possamos demonstrar alguns resultados no decorrer do trabalho. Na sequência faremos uma introdução à teoria dos jogos clássica, apresentando algumas definições e notações que serão relevantes ao trabalho, como equilíbrio de Nash, matrizes de pagamento, estratégias dominadas e estratégias mistas. Em seguida, vamos introduzir os jogos evolucionários e a dinâmica de replicação, além de definir estados evolucionariamente estáveis e equilíbrio de Nash para esse tipo de jogo para que possamos desenvolver a dinâmica de replicação em redes finitas no capítulo 5. Logo após, vamos definir o conceito de hiper-racionalidade e, na sequência, introduzir os jogos com agentes hiper-racionais e discutir a importância da assimetria de informação no jogo hiper-racional. No capítulo 8 vamos usar o pagamento de agentes hiper-racionais para definir a equação de replicação hiper-racional em redes finitas e demonstrar algumas propriedades importantes da equação. Em seguida, iremos analisar o dilema do prisioneiro hiper-racional, explorando as condições para o surgimento da cooperação, e o jogo dos colegas de quarto, que mostra como a hiper-racionalidade insere novos comportamentos interessantes na dinâmica do replicador e, logo após, iremos fazer uma comparação entre o modelo racional e o hiper-racional através de algumas simulações em diferentes tipos de grafo estrela. Por fim, iremos resumir os resultados e evidenciar oportunidades para trabalhos futuros. A implementação do modelo usada para fazer todas as simulações presentes neste trabalho pode ser acessada em \cite{Rafael2021}.
