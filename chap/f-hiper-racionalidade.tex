\chapter{Hiper-racionalidade} \label{chap:hiperRacionalidade}

Um ponto central da teoria dos jogos clássica é a hipótese de racionalidade. É através dela que os indivíduos são capazes de escolher a estratégia que mais lhe beneficiam e que dilemas interessantes, como o dilema do prisioneiro, surgem. Nos jogos evolucionários nós assumimos que cada indivíduo carrega um fenótipo que é transmitido aos seus descendentes, não havendo a hipótese de racionalidade. Já nos jogos multipopulação, ou em grafos, temos um comportamento análogo aos jogos com $N$ jogadores da teoria clássica, onde uma população, ou jogador vértice, maximiza sua utilidade através do processo evolutivo, imitando a racionalidade.

\begin{definition}[Webb \cite{webb2007game}]
    \label{defRacionalidade}
    Um indivíduo é dito racional se, dado um conjunto de possíveis resultados $\Omega=\{\omega_1,\omega_2,\omega_3,\dots\}$, ele possui preferências que satisfazem as seguintes condições:
    \begin{itemize}
        \item[1)] (Completude) Temos $\omega_1\preceq\omega_2$ ou $\omega_2\preceq\omega_1$;
        \item[2)] (Transitividade) Se $\omega_1\preceq\omega_2$ e $\omega_2\preceq\omega_3$, então $\omega_1\preceq\omega_3$.
    \end{itemize}
\end{definition}

Diversos resultados de economia experimental mostraram que a hipótese de racionalidade não descreve bem o comportamento humano. Por exemplo, muitas vezes um indivíduo irá preferir $A$ ao invés de $B$, $B$ ao invés de $C$ e $C$ ao invés de $A$ \cite{sep-game-evolutionary}. Essa falha na transitividade não ocorreria se o indivíduo possuísse um conjunto de preferências bem definido e consistente. Para descrever melhor comportamentos humanos como altruísmo, devoção, desconfiança e ciúmes, podemos considerar duas classes de preferências: pessoal e social. A preferência pessoal busca lucro ou prejuízo para si mesmo, enquanto a preferência social trata do lucro ou prejuízo para pelo menos um dos demais indivíduos, assim um indivíduo poderá escolher entre sua preferência pessoal, social ou ambas. Com isso, um jogador terá um novo ranqueamento de preferências, chamado de hiper-preferências \cite{askari2019behavioral}. Formalmente, definimos $\Omega_v=\{\omega_1,\omega_2,\dots,\omega_v,\dots,\omega_N\}$ como um conjunto de escolhas do indivíduo $v$ para a interação com cada indivíduo $i\in\{1,2,\dots,N\}$, tomando dois conjuntos de escolhas $\Omega$ e $\Omega'$, podemos definir as hiper-preferências do indivíduo como segue.

\begin{definition}[Askari, Gordji e Park \cite{askari2019behavioral}]
    \label{defHiperPreferencias}
    As hiper-preferências do jogador $v$ são uma relação sobre o conjunto de todas as escolhas do indivíduo na qual, dados $\Omega_v$ e $\Omega_v'$, temos $\Omega_v\preceq\Omega_v'$ se, para todo $u\in\{1,2,\dots,N\}$, $\omega_u\preceq\omega_u'$ ou $\omega_u'\preceq\omega_u$ baseado na preferência pessoal, social ou ambas de $v$.
\end{definition}

Note que essa relação não é necessariamente completa nem transitiva, porém um indivíduo, quando forçado a escolher entre dois conjuntos de escolhas, sempre terá predileção por um. Essa predileção depende nas condições da interação, comportamento, crenças e quaisquer outros valores que o indivíduo venha a ter. Por conta disso, as hiper-preferências nos ajudam a obter condições mais realistas para a tomada de decisão do indivíduo \cite{askari2019behavioral}.

\begin{definition}[Askari, Gordji e Park \cite{askari2019behavioral}]
    Um indivíduo é dito hiper-racional se ele é racional (\ref{defRacionalidade}) e suas hiper-preferências satisfazem ao menos uma das seguintes condições:
    \begin{itemize}
        \item [1)] O indivíduo faz sua escolha baseado em sua preferência pessoal;
        \item [2)] O indivíduo faz sua escolha baseado em sua preferência social.
    \end{itemize}
\end{definition}

Perceba que todo indivíduo hiper-racional é racional, mas nem todo indivíduo racional é hiper-racional. Podemos concluir que um indivíduo é hiper-racional se ele considera o pagamento dos demais envolvidos em suas interações, portanto ele nem sempre é capaz de identificar qual ação irá maximizar seu benefício, mas pode fazer uma escolha que maximiza o benefício dos demais agentes envolvidos. Um agente hiper-racional escolhe sua ação baseado em suas hiper-preferencias, apesar de não ter conhecimento das hiper-preferencias dos demais, ou seja, agentes hiper-racionais não possuem informação completa. Essa característica é muito importante para a dinâmica entre jogadores hiper-racionais e será mais explorada nas próximas seções.

Como dito anteriormente, a hiper-racionalidade permite que alguns comportamentos humanos possam ser expressados por agentes hiper-racionais. Um agente altruísta busca maximizar o benefício de todos os envolvidos, enquanto um agente devoto a um outro indivíduo busca maximizar o benefício deste indivíduo, mesmo que isso diminua o seu benefício pessoal, e um agente ciumento, ou invejoso, busca diminuir o benefício de agentes bem sucedidos quando ele não consegue aumentar o seu próprio.

Agora que temos o conceito de escolha hiper-racional definido nosso objetivo é aplicá-lo como base da teoria de jogos e adaptar a equação de replicação para agentes hiper-racionais. Essa construção e suas consequências serão apresentadas nos próximos capítulos.
