\chapter{Jogos com agentes hiper-racionais}

Sabemos que agentes hiper-racionais podem ter como objetivo o lucro ou prejuízo para si ou qualquer outro jogador envolvido. Para quantificar essas hiperpreferências e aplicá-las no pagamento do jogador nós iremos definir a matriz de preferências $\Rc$, onde a entrada $r_{v,u}\in[-1,1]$ denota a preferência do jogador $v$ pelo jogador $u$ e $r_{v,v}\in[-1,1]$ é preferência de $v$ por si mesmo. Quando $r_{v,u}$ é positivo, o jogador $v$ busca aumentar o lucro do jogador $u$. Quando $r_{v,u}$ é negativo, o jogador $v$ busca diminuir o lucro do jogador $u$. Se $r_{v,u}$ é zero, o jogador $v$ é indiferente quanto ao lucro ou prejuízo de $u$. Além disso, vamos definir que $\Rc$ é linha estocástica em módulo, ou seja, temos que
\begin{equation}
    \sum_{u=1}^N |r_{v,u}| = 1
\end{equation}

Essa hipótese não implica em perda de generalidade, pois, se não fosse, bastaria dividir as entradas de $\Rc$ por essa soma para tornar $\Rc$ linha estocástica em módulo.

Em outras palavras, o pagamento de um jogador hiper-racional é uma média ponderada entre o seu pagamento efetivo e um pagamento virtual proveniente de suas preferências, onde os pesos são as entradas da linha correspondente ao jogador na matriz $\Rc$. Com isso, dado um perfil de estratégias $\BS{X}=\{\BS{x_1},\BS{x_2},\dots,\BS{x_N}\}\in\Delta$ e as matrizes de pagamento $B_v,\forall v\in\V$, podemos definir o pagamento para agentes hiper-racionais como segue.

\begin{equation}
    \label{defPgtoHR}
    \pi^{\mathcal{H}}_v(\BS{X})=r_{v,v}\left[\BS{x_v} B_v \left(\frac{1}{N-1}\sum_{\substack{u=1\\u\neq v}}^N \BS{x_u}\right)\right] + \sum_{\substack{u=1\\u\neq v}}^N r_{v,u} \left(\BS{x_u} B_u \BS{x_v}\right)
\end{equation}

Temos agora um novo termo sendo somado na nossa função de pagamento, é o pagamento de $u$ no jogo clássico entre $u$ e $v$. Dizemos que esse termo é o pagamento de $u$ percebido por $v$ e vamos denotá-lo por

\begin{equation}
    \label{defPgtoPercebido}
    \pi^{\mathcal{H}}_{u,v}(\BS{x_u},\BS{x_v}) = \BS{x_u} B_u \BS{x_v}
\end{equation}

Note que o jogador $v$ não possui informação sobre as preferências de $u$, pois conhece apenas a estratégia mista e a matriz de pagamentos de $u$, além disso, para que $v$ possa estimar o impacto das suas ações no pagamento de $u$, ele assume que $u$ é um jogador racional. Essa assimetria de informação é importante e suas consequências serão discutidas na seção a seguir. 

Definiremos agora o hiperequilíbrio, que usa o pagamento $\eqref{defPgtoHR}$. Esse equilíbrio é análogo ao equilíbrio de Nash, definido em \ref{DefEqNashClassico}, porém usa o pagamento hiper-racional e tem as hiperpreferências do jogador imbuída na sua formulação. Essa distinção permite analisar separadamente os novos equilíbrios introduzidos pelas hiperpreferências dos jogadores \cite{askari2019behavioral}.

\begin{definition}
    \label{defHiperEq}
    Dado um perfil de estratégias $\BS{X^*}=(\BS{x}^*_{1},\BS{x}^*_{2},\dots,\BS{x}^*_{N})\in\Delta$ e uma matriz de preferências $\Rc$, dizemos que $\BS{X^*}$ é um hiperequilíbrio se
    \begin{equation*}
        \pi^{\mathcal{H}}_v(\BS{x}^*_v,\BS{X}^*_{-i})\geq \pi^{\mathcal{H}}_v(\BS{x}_v,\BS{X}^*_{-i})
    \end{equation*}
    para todo $\BS{x}_v\in\Delta_{M_i}$. Se a desigualdade for estrita, dizemos que $\boldsymbol{X^*}$ é um hiperequilíbrio estrito.
\end{definition}

%\begin{definition}[\citeauthor{askari2019behavioral}, 2019]
%    Um perfil de estratégias $\BS{X^*}$ é um hiper equilíbrio se, para todo jogador $v\in\V$ e todo $x_v\in\Delta_M$, temos
%    \begin{equation}
%        \pi^{\mathcal{H}}_{u,v}(\BS{x^*_u},\BS{x^*_v}) \geq \pi^{\mathcal{H}}_{u,v}(\BS{x^*_u},\BS{x_v})
%    \end{equation}
%    para todo jogador $u\in\V,u\neq v$ tal que $r_{v,u}>0$ e
%    \begin{equation}
%        \pi^{\mathcal{H}}_{u,v}(\BS{x^*_u},\BS{x^*_v}) \leq \pi^{\mathcal{H}}_{u,v}(\BS{x^*_u},\BS{x_v})
%    \end{equation}
%    para todo jogador $u\in\V,u\neq v$ tal que $r_{v,u}<0$.
%\end{definition}
%Em outras palavras, um hiper equilíbrio é um estado no qual o jogador não consegue alterar unilateralmente sua estratégia para aumentar ou diminuir o pagamento dos jogadores presentes em suas preferências. Nem todo jogo hiper-racional possui um hiper equilíbrio e nem todo hiper equilíbrio é também um equilíbrio de Nash. Veremos mais sobre isso com exemplos nos próximos capítulos.

Na próxima seção iremos analisar um jogo que possui um hiper equilíbrio diferente do equilíbrio de Nash esperado por um jogador racional, mostrando na prática o motivo de haver essa separação nas definições.

% -- % -- % -- %

\section{Assimetria de informação no jogo hiper-racional}

Num primeiro momento, pode-se pensar que as preferências sociais podem ser descritas através de mudanças na matriz de pagamentos do jogador, porém o exemplo a seguir mostra que isso não é possível, pois um jogador não possui informação sobre as preferências dos demais. Considere um jogo com dois jogadores hiper-racionais, $v_1$ e $v_2$, com a matriz de pagamentos abaixo.
\begin{table}[h]
\begin{center}
    \begin{tabular}{ccccc}
        & & \multicolumn{3}{c}{$v_2$} \\
        & & $X$ & $Y$ & $Z$ \\ \cline{3-5} 
        \multirow{3}{*}{$v_1$} & \multicolumn{1}{c|}{$A$} & \multicolumn{1}{l|}{$(1,3)$} & \multicolumn{1}{l|}{$(2,4)$} & \multicolumn{1}{l|}{$(0,6)$} \\ \cline{3-5} 
        & \multicolumn{1}{c|}{$B$} & \multicolumn{1}{l|}{$(2,2)$}  & \multicolumn{1}{l|}{$(2,2)$} & \multicolumn{1}{l|}{$(1,1)$}  \\ \cline{3-5} 
        & \multicolumn{1}{l|}{$C$} & \multicolumn{1}{l|}{$(1,1)$}  & \multicolumn{1}{l|}{$(1,1)$} & \multicolumn{1}{l|}{$(0,2)$} \\ \cline{3-5} 
    \end{tabular}
    \caption{Matriz de pagamentos do jogo exemplo.}
    \label{JogoHRassimetrico}
\end{center}
\end{table}

O jogador $v_1$ é altruísta, ou seja, ele se importa com o pagamento do seu adversário, mesmo que resulte em um pagamento efetivo menor para si próprio, porém $v_2$ age como um jogador racional clássico, se importando apenas com seu pagamento. A matriz de preferências é como segue.
\begin{equation}
    \label{matrizPrefAssimetrico}
    \mathcal{R}=
    \begin{bmatrix}
        \frac{1}{2} & \frac{1}{2}\\ 
        0 & 1 
    \end{bmatrix}
\end{equation}

Assim, enquanto $v_2$ joga usando a matriz \eqref{JogoHRassimetrico}, $v_1$ joga usando uma matriz de pagamentos virtual que leva em consideração suas preferências. O pagamento virtual de $v_1$ é dado pela matriz abaixo, onde seu pagamento foi calculado de acordo com a equação \eqref{defPgtoHR}.

\begin{table}[h]
\begin{center}
    \begin{tabular}{ccccc}
        & & \multicolumn{3}{c}{$v_2$} \\
        & & $X$ & $Y$ & $Z$ \\ \cline{3-5} 
        \multirow{3}{*}{$v_1$} & \multicolumn{1}{c|}{$A$} & \multicolumn{1}{l|}{$(2,3)$} & \multicolumn{1}{l|}{$(3,4)$} & \multicolumn{1}{l|}{$(3,6)$} \\ \cline{3-5} 
        & \multicolumn{1}{c|}{$B$} & \multicolumn{1}{l|}{$(2,2)$}  & \multicolumn{1}{l|}{$(2,2)$} & \multicolumn{1}{l|}{$(1,1)$}  \\ \cline{3-5} 
        & \multicolumn{1}{l|}{$C$} & \multicolumn{1}{l|}{$(1,1)$}  & \multicolumn{1}{l|}{$(1,1)$} & \multicolumn{1}{l|}{$(1,2)$} \\ \cline{3-5} 
    \end{tabular}
    \caption{Matriz de pagamentos virtual de $v_1$.}
    \label{PagVirtualV1}
\end{center}
\end{table}

Do ponto de vista de $v_2$, podemos ver que a estratégia $B$, de $v_1$, domina fracamente todas as demais estratégias e, portanto, $v_2$ terá de decidir entre as estratégias $X$ e $Y$, pois são as que lhe dão o maior retorno contra $B$.

\begin{table}[h]
\begin{center}
    \begin{tabular}{cccc}
        & & \multicolumn{2}{c}{$v_2$} \\
        & & $X$ & $Y$ \\ \cline{3-4} 
        $v_1$ & \multicolumn{1}{c|}{$B$} & \multicolumn{1}{l|}{$(2,2)$}  & \multicolumn{1}{l|}{$(2,2)$}  \\ \cline{3-4} 
    \end{tabular}
    \caption{Matriz após a remoção das estratégias dominadas.}
\end{center}
\end{table}

Finalmente, apesar de ambas opções resultarem no mesmo pagamento, $v_2$ irá jogar $Y$, pois possui um melhor pagamento esperado contra $A$ ou $C$. Do ponto de vista de $v_1$, a estratégia $A$ domina fracamente todas as demais e, portanto, será a sua escolha. Assim, o hiperequilíbrio desse jogo é dado pelo perfil $(A,Y)$, com pagamento efetivo $(2,4)$, enquanto os equilíbrios de Nash, esperados por $v_2$, são dados por $(B,Y)$ e $(B,X)$, com pagamento efetivo $(2,2)$.

Além disso, se as preferências pudessem ser simplesmente traduzidas na matriz de pagamento o equilíbrio seria dado pelo perfil $(A,Z)$, pois $v_2$ iria explorar o altruísmo de $v_1$ para obter um pagamento maior para si mesmo. Esse comportamento só acontece por conta da assimetria de informação entre os jogadores. Cada jogador possui um pagamento virtual próprio que influencia suas decisões e, como vimos no exemplo acima, muda o resultado do jogo. Porém, em um jogo iterado, $v_2$ seria capaz de, eventualmente, descobrir e explorar o altruísmo de $v_1$. Esse comportamento é esperado de um jogador racional, que busca o melhor pagamento para si, mas ainda assim o resultado seria diferente dos equilíbrios de Nash do jogo.
