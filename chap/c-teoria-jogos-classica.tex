\chapter{Teoria dos Jogos Clássica}

Tomar uma decisão importante não é tarefa fácil. Para adotar a estratégia que nos trará o melhor resultado devemos considerar os possíveis cenários e como nossa decisão afeta os demais agentes envolvidos. Na Teoria dos Jogos, o processo descrito acima é chamado de jogo e os agentes envolvidos nele são chamados de jogadores. Cada jogador possui um conjunto de estratégias, do qual ele deve escolher qual usar. Um jogador é dito racional se ele é capaz de ordenar os resultados do jogo de acordo com suas preferências e, então, escolher a que lhe traz maior benefício. No momento em que todos os jogadores têm suas estratégias definidas, teremos um perfil de estratégias. Cada jogador pode possuir seus objetivos próprios e, portanto, diferentes recompensas para cada possível perfil de estratégias. Chamamos de pagamento, ou payoff, a recompensa de cada jogador.

Um exemplo clássico de jogo é o Dilema do Prisioneiro, que mostra como o conflito de interesses afeta a tomada de decisão do indivíduo e o resultado final do grupo. Nele, dois prisioneiros são capturados e separados. A eles são dadas duas opções: ficar calado ou colaborar com a polícia, fazendo um acordo de colaboração premiada. No nosso exemplo, se ambos ficarem em silêncio, pegarão uma pena de dois anos de prisão, pois a polícia não possui provas para acusá-los de todos os crimes. Se um ficar em silêncio e o outro assinar o acordo, então o delator sairá livre, enquanto o delatado pegará uma pena de sete anos de prisão. Caso ambos colaborem com a polícia, ambos pegarão uma pena de cinco anos de prisão.

Apesar da aparente simplicidade, o Dilema do Prisioneiro pode ser usado para modelar diversas situações onde há competição ou cooperação entre indivíduos. A seguir, iremos definir os conceitos necessários para se analisar, e resolver, matematicamente os diferentes tipos de jogos, entre eles o Dilema do prisioneiro.

%A melhor opção para ambos é o silêncio, onde os dois ficariam apenas dois anos na prisão, mas essa não é a melhor solução possível para o indivíduo. Se o prisioneiro decidir pela delação, ele sairá livre e, ao mesmo tempo, garante uma pena menor caso o seu parceiro também decida pela delação. Portanto, mesmo sendo o pior cenário possível para o grupo, a delação mútua será o resultado final do Dilema do Prisioneiro.

% -- % -- % -- %

\section{Noções gerais}

Para definir um jogo, precisamos primeiro de um conjunto de jogadores, seus conjuntos de estratégias e o pagamento de cada perfil de estratégias para cada jogador. Formalmente, temos um conjunto finito de jogadores $\V=\{v_1,v_2,\dots,v_N\}$ e cada jogador possui um conjunto de estratégias $S_i=\{s_{i,1},s_{i,2},\dots,s_{i,M_i}\},1\leq i\leq N$. Assim, o conjunto de todos os perfis de estratégias, chamado de espaço de estratégias do jogo, é dado pelo produto cartesiano
\begin{equation}
\label{defS}
    \boldsymbol{S}=S_1\times S_2\times\dots\times S_N.
\end{equation}
Um vetor $\boldsymbol{s}=[ s_1 \; \cdots \; s_N]$, onde $s_i\in S_i$, é um perfil de estratégias do jogo. Note que $s_i=s_{i,j}$, para algum $j\in\{1,2,\dots,M_i\}$, denota a estratégia pura escolhida pelo jogador $i$ dentre as $M_i$ disponíveis. Finalmente, cada jogador possui sua própria função de pagamentos 
\begin{align}
\label{defU}
\begin{split}
    u_i \colon \boldsymbol{S} &\to \R\\
    s_i &\mapsto u_i(\boldsymbol{s})
\end{split}
\end{align}
que associa uma recompensa ao jogador $v_i$ para cada perfil de estratégias ${\boldsymbol{s}\in \boldsymbol{S}}$.

Voltando ao nosso exemplo, temos $\V=\{v_1,v_2\}$, os dois prisioneiros, e seus conjuntos de estratégias $S_1=S_2=\{\textit{silêncio, delação}\}$. Assim, definimos a função de pagamento $u_{v_1}\colon\boldsymbol{S}\to\R$ de $v_1$ como
\begin{align}
\label{pagp1}
\begin{split}
    u_{v_1}(\textit{silêncio, silêncio})&=-2 \\
    u_{v_1}(\textit{silêncio, delação})&=-7 \\
    u_{v_1}(\textit{delação, silêncio})&=0 \\
    u_{v_1}(\textit{delação, delação})&=-5
\end{split}
\end{align}
e a função de pagamento $u_{v_2}\colon\boldsymbol{S}\to\R$ de $v_2$ como
\begin{align}
\label{pagp2}
\begin{split}
    u_{v_2}(\textit{silêncio, silêncio})&=-2 \\
    u_{v_2}(\textit{silêncio, delação})&=0 \\
    u_{v_2}(\textit{delação, silêncio})&=-7 \\
    u_{v_2}(\textit{delação, delação})&=-5
\end{split}
\end{align}
Note que podemos expressar essas funções através de uma matriz, que chamamos de matriz de pagamentos do jogo.

\begin{table}[h]
\begin{center}
    \begin{tabular}{cccc}
        & & \multicolumn{2}{c}{$v_2$} \\
        & & Silêncio & Delação      \\ \cline{3-4} 
        \multirow{2}{*}{$v_1$} & \multicolumn{1}{c|}{Silêncio} & \multicolumn{1}{c|}{$(-2,-2)$} & \multicolumn{1}{c|}{($-7,0)$}  \\ \cline{3-4} 
        & \multicolumn{1}{c|}{Delação} & \multicolumn{1}{c|}{$(0,-7)$}  & \multicolumn{1}{c|}{$(-5,-5)$} \\ \cline{3-4} 
    \end{tabular}
    \caption{Matriz de pagamentos do dilema do prisioneiro.}
    \label{mpdp}
\end{center}
\end{table}


A matriz nos dá o pagamento de ambos os jogadores em cada uma das situações possíveis, em cada posição $i,j$ da matriz, a primeira entrada do vetor é o pagamento do jogador 1 ao jogador $i$ contra o jogador 2 jogando $j$ e a segunda entrada é o pagamento do jogador 2 na mesma situação. Nesse caso, a matriz de pagamentos é idêntica para ambos jogadores e, portanto, dizemos que o jogo é simétrico. Esse exemplo mostra o tipo mais simples de jogo, pois temos apenas dois jogadores, o pagamento é simétrico e os jogadores possuem informação completa, ou seja, todos sabem quais as estratégias e função de pagamento de todos jogadores. Além disso, dizemos que um jogo representado em uma matriz de pagamentos, como na Tabela \ref{mpdp}, está em sua forma normal.

% -- % -- % -- %

\section{Soluções de um jogo}

Uma solução é um perfil de estratégias que maximiza dinamicamente o pagamento de todos jogadores, ou seja, é o perfil que, dado o contexto, dá o maior payoff possível para todos. A solução é uma previsão do que irá acontecer, dada a racionalidade dos jogadores, e descreve quais estratégias serão utilizadas por cada jogador e, portanto, o resultado do jogo. Neste trabalho iremos ver duas das maneiras mais utilizadas de solução de um jogo, dominância e Equilíbrio de Nash.

No Dilema do Prisioneiro, a melhor opção para o grupo seria o silêncio, onde os dois ficariam apenas dois anos na prisão, mas esse não é o melhor resultado possível para o indivíduo. Se um prisioneiro decidir pela delação, ele poderá sair livre se seu parceiro ficar em silêncio e, ao mesmo tempo, garante uma pena menor caso ele também decida pela delação. Portanto, quando assumimos a racionalidade dos jogadores, mesmo sendo o pior cenário possível para o grupo, a delação mútua será a solução do Dilema do Prisioneiro.

Esse raciocínio pode ser visualizado na tabela de pagamentos através de um processo chamado Dominância Estrita Iterada. Para descrevê-lo melhor, primeiro precisamos de algumas definições. Queremos analisar a escolha estratégica do jogador $v_i\in V$, então definimos, por comodidade,
\begin{equation*}
    s_{-i}=\{s_{1},s_{2},\dots,s_{i-1},s_{i+1},\dots,s_{N}\}
\end{equation*}
como o conjunto de estratégias para todos jogadores, menos $v_i$. Da mesma forma, definimos $\boldsymbol{S_{-i}}=S_1\times S_2\times\dots\times S_{i-1}\times S_{i+1}\times\dots\times S_N$, tal que $s_{-i}\in \boldsymbol{S_{-i}}$. Com isso, dado $v_i$, podemos escrever um perfil de estratégias de um jogo convenientemente por
\begin{equation*}
    \boldsymbol{s}=(s_{i},s_{-i})=\{s_{1},s_{2},\dots,s_{i-1},s_{i},s_{i+1},\dots,s_{N}\}
\end{equation*}
Por fim, podemos definir o que é uma estratégia pura dominada.
\begin{definition}
    Uma estratégia pura $s_{i}\in S_i$ do jogador $v_i\in V$ é dita fracamente dominada pela estratégia $s'_{i}\in S_i$ quando
    \begin{equation}
        u_i(s'_{i},s_{-i})\geq u_i(s_{i},s_{-i}), \forall s_{-i}\in \boldsymbol{S_{-i}}.
        \label{fracaDominado}
    \end{equation}
    Se a desigualdade é estrita, dizemos que a estratégia $s_{i}$ é estritamente dominada pela estratégia $s'_{i}$. Além disso, dizemos que uma estratégia $s'_i\in S_i$ é uma estratégia dominante se todas estratégias $s_{i}\in S_i$ do jogador $v_i\in V$ são fracamente dominadas por ela.
\end{definition}

A desigualdade $\eqref{fracaDominado}$ nos diz que jogar $s'_{i}$ é sempre melhor ou igual do que jogar $s_i$, não importa qual seja a estratégia escolhida pelos demais jogadores e, portanto, $v_i$ jamais jogará $s_i$.

Dominância Estrita Iterada é o processo de eliminar as estratégias estritamente dominadas iteradamente para obter o resultado de um jogo. Para exemplificar esse conceito, considere a matriz de pagamentos abaixo.

\begin{table}[h]
\begin{center}
    \begin{tabular}{ccccc}
        & & \multicolumn{3}{c}{$v_2$} \\
        & & X & Y & Z \\ \cline{3-5} 
        \multirow{3}{*}{$v_1$} & \multicolumn{1}{c|}{A} & \multicolumn{1}{l|}{$(\colorbox{LimeGreen}{3},4)$} & \multicolumn{1}{l|}{$(\colorbox{LimeGreen}{5},3)$} & \multicolumn{1}{l|}{$(\colorbox{LimeGreen}{9},2)$} \\ \cline{3-5} 
        & \multicolumn{1}{c|}{B} & \multicolumn{1}{l|}{$(0,1)$}  & \multicolumn{1}{l|}{$(4,6)$} & \multicolumn{1}{l|}{$(6,0)$}  \\ \cline{3-5} 
        & \multicolumn{1}{l|}{C} & \multicolumn{1}{l|}{$(\colorbox{Salmon}{2},1)$}  & \multicolumn{1}{l|}{$(\colorbox{Salmon}{3},5)$} & \multicolumn{1}{l|}{$(\colorbox{Salmon}{2},8)$} \\ \cline{3-5} 
    \end{tabular}
    \caption{Matriz de pagamentos do jogo exemplo.}
    \label{mpjdi1}
\end{center}
\end{table}
Neste jogo, a estratégia $C$ (em vermelho) é estritamente dominada pela $A$ (em verde) para o jogador $v_1$, portanto, podemos excluí-la da matriz de pagamentos, já que nunca será usada.

\begin{table}[h]
\begin{center}
    \begin{tabular}{ccccc}
        & & \multicolumn{3}{c}{$v_2$} \\
        & & X & Y & Z \\ \cline{3-5} 
        \multirow{2}{*}{$v_1$} & \multicolumn{1}{c|}{A} & \multicolumn{1}{l|}{$(3,\colorbox{LimeGreen}{4})$} & \multicolumn{1}{l|}{$(5,\colorbox{LimeGreen}{3})$} & \multicolumn{1}{l|}{$(9,\colorbox{Salmon}{2})$} \\ \cline{3-5} 
        & \multicolumn{1}{c|}{B} & \multicolumn{1}{l|}{$(0,\colorbox{LimeGreen}{1})$}  & \multicolumn{1}{l|}{$(4,\colorbox{LimeGreen}{6})$} & \multicolumn{1}{l|}{$(6,\colorbox{Salmon}{0})$}  \\ \cline{3-5} 
    \end{tabular}
    \caption{Nova matriz de pagamentos, após a remoção da estratégia $C$.}
    \label{mpjdi2}
\end{center}
\end{table}

Agora vemos que, para o jogador $v_2$, a estratégia $Z$ (em vermelho) é estritamente dominada por $X$ e $Y$ (ambas em verde) e, portanto, podemos removê-la da matriz de pagamentos.

\begin{table}[h]
\begin{center}
    \begin{tabular}{cccc}
        & & \multicolumn{2}{c}{$v_2$} \\
        & & X & Y \\ \cline{3-4} 
        \multirow{2}{*}{$v_1$} & \multicolumn{1}{c|}{A} & \multicolumn{1}{l|}{$(3,4)$} & \multicolumn{1}{l|}{$(5,3)$} \\ \cline{3-4}
        & \multicolumn{1}{c|}{B} & \multicolumn{1}{l|}{$(0,1)$}  & \multicolumn{1}{l|}{$(4,6)$} \\ \cline{3-4} 
    \end{tabular}
    \caption{Nova matriz de pagamentos, após a remoção da estratégia $Z$.}
    \label{mpjdi3}
\end{center}
\end{table}

Seguindo este processo, vemos que a estratégia $B$ é dominada pela estratégia $A$ para o jogador $v_1$ e, finalmente, que a estratégia $Y$ é dominada por $X$ para o jogador $v_2$. Isso implica que o resultado do jogo será $(3,4)$, pois resta apenas a estratégia $A$, para $v_1$, e $X$, para $v_2$. Note que haviam resultados melhores para ambos jogadores, como $(4,6)$, e resultados melhores para cada indivíduo, como $(9,2)$ e $(2,8)$, porém, devido à racionalidade dos jogadores, o resultado final não é o melhor para o grupo, nem mesmo para o indivíduo.

Neste exemplo, ao eliminar as estratégias estritamente dominadas fomos conduzidos a um único perfil de estratégias como solução do jogo. Porém, nem sempre esse processo nos fornecerá apenas um perfil, na maioria dos casos iremos obter vários perfis ou até mesmo todo o espaço de estratégias como, por exemplo, no jogo de Pedra, Papel e Tesoura. Como sabemos, Pedra ganha de Tesoura, que ganha do Papel, que ganha da Pedra. Assim, como podemos ver na matriz de pagamentos abaixo, nenhuma estratégia domina as demais e, portanto, nenhuma estratégia será eliminada ao utilizar a Dominância Estrita Iterada.
\begin{table}[h]
\begin{center}
    \begin{tabular}{ccccc}
        & & & $v_2$ & \\
        & & Pe & Pa & T \\ \cline{3-5} 
        & \multicolumn{1}{c|}{Pe} & \multicolumn{1}{c|}{(0,0)}  & \multicolumn{1}{c|}{(-1,1)} & \multicolumn{1}{c|}{(1,-1)} \\ \cline{3-5} 
        $v_1$ & \multicolumn{1}{c|}{Pa} & \multicolumn{1}{c|}{(1,-1)} & \multicolumn{1}{c|}{(0,0)}  & \multicolumn{1}{c|}{(-1,1)} \\ \cline{3-5} 
         & \multicolumn{1}{c|}{T}  & \multicolumn{1}{c|}{(-1,1)} & \multicolumn{1}{c|}{(1,-1)} & \multicolumn{1}{c|}{(0,0)}  \\ \cline{3-5} 
    \end{tabular}
    \caption{Matriz de pagamentos do jogo Pedra, Papel e Tesoura.}
    \label{mppepat}
\end{center}
\end{table}

\FloatBarrier % Para a tabela não ficar no meio da definição

Para casos como esse nós iremos precisar do outro conceito citado no início desta seção, o Equilíbrio de Nash, que é formalmente definido a seguir.

\begin{definition}
    Dizemos que o perfil de estratégias
    \begin{equation*}
        \boldsymbol{s^*}=\{s^*_1,\dots,s^*_{i-1},s^*_i,s^*_{i+1},\dots,s^*_N\}
    \end{equation*}
    é um equilíbrio de Nash se
    \begin{equation}
        u_i(s^*_i,s^*_{-i})\geq u_i(s_i,s^*_{-i})
        \label{defEqNash}
    \end{equation}
    para todo $i=1,\dots,N$ e $s_i\in S_i$. Se a desigualdade for estrita, dizemos que $\boldsymbol{s^*}$ é um equilíbrio de Nash estrito.
\end{definition}

De maneira mais geral, a equação $\eqref{defEqNash}$ diz que $v_i$ não é capaz de melhorar seu pagamento ao trocar, de maneira unilateral, $s^*_i$ por qualquer outra estratégia. Assim, vemos que uma solução é um perfil de estratégias no qual nenhum jogador tem incentivos para mudar de estratégia. Dizemos que um perfil de estratégias com essa característica é um Equilíbrio de Nash. Observe que toda estratégia dominante é um equilíbrio de Nash, mas nem todo equilíbrio de Nash é uma estratégia dominante.

Com isso concluímos que o único equilíbrio de Nash do Dilema do Prisioneiro é o perfil $(\textit{delação, delação})$ e no jogo dado pela Tabela \ref{mpjdi1} o único equilíbrio de Nash é dado pelo perfil $(A,X)$. Porém, no Pedra, Papel e Tesoura, vemos que não há um equilíbrio de Nash em estratégias puras, pois um dos jogadores sempre pode melhorar seu pagamento dependendo da estratégia do seu adversário.

% -- % -- % -- %

\section{Estratégias Mistas}

Como vimos acima, um jogo nem sempre possui um equilíbrio de Nash em estratégias puras. Para contornar isso, podemos avaliar o jogo do ponto de vista probabilístico, onde o jogador adotará uma distribuição de probabilidades sobre suas estratégias puras e escolherá a estratégia que irá jogar de acordo com essa distribuição. 

\begin{definition}
\label{eqn}
    Uma estratégia $\boldsymbol{x}_i$ do jogador $v_i\in V$ é dita uma estratégia mista quando $\boldsymbol{x}_i$ é uma distribuição de probabilidades sobre o conjunto $S_i$, ou seja, $\boldsymbol{x}_i$ é um elemento do conjunto
    \begin{equation}
        \Delta_{M_i}=\left\{(x_{i,1},\dots,x_{i,M_i})\in\R^{M_i}\;\mid\;  x_{i,k}\geq 0, \sum_{k=1}^{M_i} x_{i,k}=1\right\}.
    \end{equation}
    Além disso, espaço de todas as estratégias mistas é dado pelo produto cartesiano
    \begin{equation}
        \label{defDelta}
        \Delta=\Delta_{M_1}\times\dots\times\Delta_{M_N}
    \end{equation}
    e é chamado de Espaço de Estratégias Mistas.
\end{definition}

Assim, dizemos que um vetor $\boldsymbol{X}\in\Delta$ é um perfil de estratégia mista para o jogo. Também iremos estender a notação utilizada em perfis puros para os perfis mistos, portanto dizemos que $\boldsymbol{X}_{-i}$ é um conjunto de estratégias mistas para todos os jogadores, menos $v_i$.

O pagamento esperado em estratégias mistas é dado por uma média dos pagamentos puros ponderada pelas probabilidades. Assim, dado um conjunto $\V$ de $N$ jogadores, onde cada jogador $i\in\V$ possui sua função de pagamento $u_i$, um perfil de estratégias $\BS{X}\in\Delta$, tal que
\begin{equation*}
    \boldsymbol{X}=(\boldsymbol{x}_1,\dots,\boldsymbol{x}_N),
\end{equation*}
que é equivalente à
\begin{equation*}
    \boldsymbol{X}=(x_{1,1},x_{1,2},\dots,x_{1,M_1},\dots,x_{N,1},x_{N,2},\dots,x_{N,M_N}).
\end{equation*}
definimos o pagamento esperado de $v$ ao usar a estratégia mista $x_i$ como
\begin{equation}
    \label{defPgtoClassico}
    u_i(\BS{X})=\sum_{j_1=1}^{M_1}\sum_{j_2=1}^{M_2}\dots\sum_{j_N=1}^{M_N}
                    \left( \prod_{k=1}^N x_{k,j_k}u_i(s_{1,j_1},s_{2,j_2},\dots,s_{N,j_N})\right)
\end{equation}

A definição de Equilíbrio de Nash para estratégias mistas é análoga à definição $\eqref{eqn}$, sendo reescrita a seguir.

\begin{definition}
    \label{DefEqNashClassico}
    Dizemos que o perfil de estratégias
    \begin{equation*}
        \BS{X^*}=(\BS{x}^*_{1},\BS{x}^*_{2},\dots,\BS{x}^*_{N})\in\Delta
    \end{equation*}
    é um equilíbrio de Nash se
    \begin{equation*}
        u_i(\BS{x}^*_i,\BS{X}^*_{-i})\geq u_i(\BS{x},\BS{X}^*_{-i})
    \end{equation*}
    para todo $\BS{x}_i\in\Delta_{M_i}$. Se a desigualdade for estrita, dizemos que $\boldsymbol{X^*}$ é um equilíbrio de Nash estrito.
\end{definition}

Com isso em mãos, podemos procurar soluções em estratégias mistas nos nossos exemplos. No dilema do prisioneiro, o perfil $\BS{s^*}=(\textit{delação, delação})$ é um equilíbrio de Nash puro. Tome uma estratégia mista $\BS{x}=(x,1-x)\in\Delta_1$, onde a primeira e segunda coordenada do vetor são, respectivamente, a probabilidade do jogador escolher a estratégia pura \textit{silêncio} e a probabilidade do jogador escolher a estratégia pura \textit{delação}, então, dado que o outro jogador não mude sua estratégia, temos
\begin{equation*}
    u_1(\BS{x},s^*_2)=-7x-(1-x)5=-2x-5\leq -5 = u_1(\BS{s^*})
\end{equation*}
Como o raciocínio é simétrico, vemos que não há equilíbrio de Nash misto e que a delação mútua é o único equilíbrio do jogo.

Vimos que no exemplo do Pedra, Papel e Tesoura não há equilíbrio de Nash puro, pois fixada uma escolha para um jogador, o outro jogador sempre poderá mudar sua estratégia para aumentar seu pagamento. O equilíbrio misto para este jogo está em utilizar cada estratégia com igual probabilidade, dado pela estratégia $(\frac{1}{3},\frac{1}{3},\frac{1}{3})$. Para verificar isso, assuma que um jogador esteja utilizando a estratégia $s^*=(\frac{1}{3},\frac{1}{3},\frac{1}{3})$ e defina $e_1=(1,0,0),e_2=(0,1,0)$ e $e_3=(0,0,1)$ como as estratégias puras Pedra, Papel e Tesoura, respectivamente, então o pagamento esperado para o outro jogador será
\begin{equation}
    \left\{\begin{matrix}
        u(e_1,s^*) & = \frac{1}{3}-\frac{1}{3} =0\\ 
        u(e_2,s^*) & = \frac{1}{3}-\frac{1}{3} =0\\ 
        u(e_3,s^*) & = \frac{1}{3}-\frac{1}{3} =0
    \end{matrix}\right.
\end{equation}

Como podemos ver, não há incentivo para que o jogador mude de estratégia de maneira unilateral e, portanto, temos um equilíbrio de Nash em $(\frac{1}{3},\frac{1}{3},\frac{1}{3})$.

Todos exemplos possuem pelo menos um equilíbrio de Nash, seja ele puro ou misto. De fato, John Nash demonstrou que todo jogo definido por matrizes de pagamento possui pelo menos um equilíbrio de Nash em estratégias mistas \cite{nash1950equilibrium}. Para mostrar esse resultado, iremos primeiro enunciar o teorema do ponto fixo de Brouwer.

\begin{theorem}[Ponto fixo de Brouwer]
\label{teoPontoFixo}
    Seja $F:E\to E$ uma função contínua, com $E$ convexo, fechado e limitado. Então $F$ possui um ponto fixo, isto é, $\exists x\in E$ tal que $F(x)=x$.
\end{theorem}

A demonstração do teorema acima pode ser encontrada em \cite{starr_2011}. Com a notação utilizada até o momento, iremos demonstrar uma série de teoremas \cite{sartini2004introduccao} que fornecem caracterizações alternativas para o equilíbrio de Nash e os usaremos para demonstrar o teorema de Nash.

\begin{theorem}
    Para cada $i=1,2,\dots,N$ e $j=1,2,\dots,M_i$, defina as funções
    \begin{align*}
        \begin{split}
            z_{i,j} \colon \Delta &\to \R\\
            \BS{X} &\mapsto z_{i,j}(\BS{X})=u_i(s_{i,j},\BS{X_{-i}})-u_i(\BS{x_i},\BS{X_{-i}})
        \end{split}
    \end{align*}
    isto é, $z_{i,j}$ mede o ganho ou perda do jogador $i$ ao trocar a estratégia mista $\BS{x_i}$ pela estratégia pura $s_{i,j}$. Temos que $\BS{X^*}$ é um equilíbrio de Nash se, e somente se,
    \begin{equation*}
        z_{i,j}(\BS{X^*})\leq 0
    \end{equation*}
    para cada $i=1,2,\dots,N$ e $j=1,2,\dots,M_i$.
\end{theorem}
\begin{proof}[Dem.:]
    Se $\BS{X^*}$ é um equilíbrio de Nash, segue da definição que
    \begin{equation}
        u_i(\BS{x}^*_i,\BS{X}^*_{-i})\geq u_i(s_{i,j},\BS{X}^*_{-i})
    \end{equation}
    para cada $i=1,2,\dots,N$ e $j=1,2,\dots,M_i$. Logo,
    \begin{multline}
        z_{i,j}(\BS{x_i})=u_i(s_{i,j},\BS{X_{-i}})-u_i(\BS{x_i},\BS{X_{-i}})\leq 0
    \end{multline}
    Por outro lado, se $z_{i,j}(\BS{x_i})\leq 0$ para cada $i=1,2,\dots,N$ e $j=1,2,\dots,M_i$. Temos,
    \begin{equation}
        u_i(s_{i,j},\BS{X_{-i}})=u_i(\BS{e_j},\BS{X_{-i}})\leq u_i(\BS{x_i},\BS{X_{-i}})
    \end{equation}
    onde $\BS{e_j}$ é o vetor unitário em $\R^{m_i}$ com $1$ na $j$-ésima posição. Queremos mostrar que, para todo $\BS{x_i}\in\Delta_{M_i}$,
    \begin{equation}
        u_i(\BS{x}^*_i,\BS{X}^*_{-i})\geq u_i(\BS{x_i},\BS{X}^*_{-i})
    \end{equation}
    Porém, como $\BS{x_i}\mapsto u_i(\BS{x_i},\BS{X^*_{-i}})$ é a média ponderada dos pagamentos em estratégias puras pela probabilidade de $i$ usar aquela estratégia pura e, portanto, um funcional linear, temos que
    \begin{equation}
    \begin{split}
        u_i(\BS{x_i},\BS{X^*_{-i}}) &= u_i\left(\sum_{j=1}^{M_i}x_{i,j}\BS{e_j},\BS{X^*_{-i}}\right)
        = \sum_{j=1}^{M_i}x_{i,j}u_i\left(\BS{e_j},\BS{X^*_{-i}}\right) \\
        & \leq\sum_{j=1}^{M_i}x_{i,j}u_i\left(\BS{x_i^*},\BS{X^*_{-i}}\right)
        = u_i\left(\BS{x_i^*},\BS{X^*_{-i}}\right),
    \end{split}
    \end{equation}
    onde, na última igualdade, usamos o fato de que  $\sum_{j=1}^{M_i}x_{i,j}=1$, pois $\BS{x_i}\in\Delta_{M_i}$
\end{proof}

\raggedbottom % Pra arrumar o espaçamento estranho

\begin{theorem}
    Para cada $i=1,2,\dots,N$ e $j=1,2,\dots,M_i$, defina as funções
    \begin{align*}
        \begin{split}
            g_{i,j} \colon \Delta &\to \R\\
            \BS{X} &\mapsto g_{i,j}(\BS{X})=\textup{max}\{0,z_{i,j}(\BS{X})\}
        \end{split}
    \end{align*}
    Temos que $\BS{X^*}$ é um equilíbrio de Nash se, e somente se,
    \begin{equation*}
        g_{i,j}(\BS{X^*})= 0
    \end{equation*}
    para cada $i=1,2,\dots,N$ e $j=1,2,\dots,M_i$.
\end{theorem}

\begin{proof}[Dem.:]
    A demonstração segue imediatamente do teorema anterior.
\end{proof}

\begin{theorem}
    Defina a aplicação
    \begin{align*}
        \begin{split}
            \BS{F} \colon \Delta &\to \Delta\\
            \BS{X} &\mapsto \BS{F}(\BS{X})=(\BS{y_1}(\BS{X}),\BS{y_2}(\BS{X}),\dots,\BS{y_N}(\BS{X})
        \end{split}
    \end{align*}
    onde $\BS{y_i}(\BS{X})=(y_{i,1}(\BS{X}),y_{i,2}(\BS{X}),\dots,y_{i,M_i}(\BS{X})),\BS{x_i}\in\Delta_{M_i}$ e
    \begin{equation*}
        y_{i,j}=\frac{x_{i,j}+g_{i,j}(\BS{X})}{1+\sum\limits_{k=1}^{M_i}g_{i,k}(\BS{X})}
    \end{equation*}
    Temos que $\BS{X^*}$ é um equilíbrio de Nash se, e somente se,
    \begin{equation*}
        \BS{F}(\BS{X^*})=\BS{X^*},
    \end{equation*}
    isto é, se, e somente se, $\BS{X^*}$ é um ponto fixo de $\BS{F}$.
\end{theorem}
\begin{proof}[Dem.:]
    Observe que, de fato, $\BS{F}(\Delta)\subseteq\Delta$, pois claramente $y_{i,j}\geq 0$ e
    \begin{equation}
        \sum_{k=1}^{M_i}y_{i,k}=\sum_{k=1}^{M_i}\left( \frac{x_{i,k}+g_{i,k}(\BS{X})}{1+\sum\limits_{j=1}^{M_i}g_{i,j}(\BS{X})} \right)=
        \frac{\sum\limits_{k=1}^{M_i}x_{i,k}+\sum\limits_{k=1}^{M_i}g_{i,k}(\BS{X})}{1+\sum\limits_{k=1}^{M_i}g_{i,k}(\BS{X})}
    \end{equation}
    usando o fato de que $\BS{x_i}\in\Delta_i$, temos
    \begin{equation}
         \sum_{k=1}^{M_i}y_{i,k}=\frac{1+\sum\limits_{k=1}^{M_i}g_{i,k}(\BS{X})}{1+\sum\limits_{k=1}^{M_i}g_{i,k}(\BS{X})} = 1
    \end{equation}
    isto é, cada $\BS{y_i}(\BS{X})\in\Delta_{M_i}$.
    
    Se $\BS{X^*}$ é um equilíbrio de Nash, então $g_{i,j}(\BS{X^*})=0$ para cada $i=1,2,\dots,N$ e $j=1,2,\dots,M_i$. Desta maneira, $y_{i,j}(\BS{X^*})=x^*_{i,j}$ para cada $i=1,2,\dots,N$ e ${j=1,2,\dots,M_i}$, isto é, $\BS{y_i}(\BS{X^*})=\BS{x_i^*},\forall i=1,2,\dots,N$, ou, ainda, $\BS{F}(\BS{X^*})=\BS{X^*}$.
    
    Agora suponha que $\BS{X^*}=(\BS{x^*_1},\BS{x^*_2},\dots,\BS{x^*_N})\in\Delta$ seja um ponto fixo de $\BS{F}$. Então,
    \begin{equation}
        x^*_{i,j}=\frac{x^*_{i,j}+g_{i,j}(\BS{X^*})}{1+\sum\limits_{k=1}^{M_i}g_{i,k}(\BS{X^*})}
    \end{equation}
    para todo $i=1,2,\dots,N$ e $j=1,2,\dots,M_i$. Segue que
    \begin{equation}
        x^*_{i,j}\sum_{k=1}^{M_i}g_{i,k}(\BS{X^*}) = g_{i,j}(\BS{X^*})
    \end{equation}
    para todo $i=1,2,\dots,N$ e $j=1,2,\dots,M_i$. Afirmamos que ${\alpha=\sum_{k=1}^{M_i}g_{i,k}(\BS{X^*})=0}$, de modo que $g_{i,k}(\BS{X^*})=0$ para todo $i=1,2,\dots,N$ e $k=1,2,\dots,M_i$. De fato, se, por absurdo, $\alpha>0$, vemos a partir da relação acima que $g_{i,j}(\BS{X^*})>0$ se, e somente se, $x^*_{i,j}>0$. Sem perda de generalidade, suponha que $x^*_{i,1}>0,x^*_{i,2}>0,\dots,x^*_{i,l}>0$ e $x^*_{i,(l+1)}=x^*_{i,(l+2)}= \dots=x^*_{i,m_i}=0$. Observe que
    \begin{equation}
        \BS{x^*_i}=\sum_{k=1}^{M_i}x_{i,k}^* \BS{e_k}
    \end{equation}
    
    Dado que $g_{i,k}(\BS{X^*})>0$ para todo $k=1,2,\dots,l$, temos que
    \begin{equation}
        u_i(\BS{e_k},\BS{X}^*_{-i})\geq u_i(\BS{x}^*_i,\BS{X}^*_{-i})
    \end{equation}
    o que nos dá
    \begin{equation}
    \begin{split}
        u_i(\BS{x}^*_i,\BS{X}^*_{-i}) &= u_i(\sum_{k=1}^{M_i}x_{i,k}^* \BS{e_k},\BS{X}^*_{-i}) = \sum_{k=1}^{l}x_{i,k}^* u_i(\BS{e_k},\BS{X}^*_{-i}) \\
        &> \sum_{k=1}^{l}x_{i,k}^* u_i(\BS{x}^*_i,\BS{X}^*_{-i}) = u_i(\BS{x}^*_i,\BS{X}^*_{-i})
    \end{split}
    \end{equation}
    um absurdo. Isto demonstra que $g_{i,j}(\BS{X^*})=0$ para todo $i=1,2,\dots,N$ e $j=1,2,\dots,M_i$ e, assim, $\BS{X^*}$ é um equilíbrio de Nash em estratégias mistas.
\end{proof}

Com isso, podemos finalmente demonstrar o teorema de Nash.

\begin{theorem}
    Todo jogo de $N$ jogadores com um conjunto finito de estratégias puras possui equilíbrio de Nash.
\end{theorem}
\begin{proof}[Dem.:]
    A aplicação $\BS{F} \colon \Delta \to \Delta$ definida no teorema anterior é contínua e $\Delta$ é um conjunto compacto e convexo. Pelo teorema do ponto fixo de Brouwer, $\BS{F}$ possui um ponto fixo $\BS{X^*}$. Pelo teorema anterior, $\BS{X^*}$ é um equilíbrio de Nash.
\end{proof}
