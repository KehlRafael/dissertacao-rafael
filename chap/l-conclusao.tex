\chapter{Considerações finais}

A introdução da hiper-racionalidade e da parte social da equação de replicação adiciona mais uma camada de complexidade na dinâmica do replicador e, como vimos, nos permite modelar comportamentos mais complexos. Inicialmente, nós definimos a função de pagamento $\eqref{defPgtoHR}$ para um jogador hiper-racional $v$, que é formada pela soma do pagamento de $v$ com o pagamento dos demais jogadores como percebido por $v$ ponderada pelas hiperpreferências de $v$. Usando esse pagamento, nós definimos o conceito de hiperequilíbrio \ref{defHiperEq}, que é análogo ao equilíbrio de Nash, usando o pagamento para jogadores hiper-racionais, o que facilita a comparação e análise dos resultados para jogadores racionais e hiper-racionais. Vimos a importância dessa separação ao analisar aos efeitos da assimetria de informação em um jogo estático, onde o hiperequilíbrio do jogo, encontrado através da dominância não estrita iterada nos pagamentos virtuais, não é um equilíbrio de Nash, pois um jogador é capaz de melhorar seu pagamento efetivo ao mudar de estratégia unilateralmente, e isso se deve à assimetria de informação do jogo com jogadores hiper-racionais.

Também analisamos o dilema do prisioneiro com jogadores hiper-racionais e mostramos através da expressão $\eqref{rSela}$ para os valores de $r_i$ que o resultado do jogo depende, principalmente, da matriz de pagamentos e das hiperpreferências do jogador. Com isso, mostramos que a cooperação é predominante dentre jogadores hiper-racionais quando consideramos preferências positivas por si e pela vizinhança.

Reescrevemos a equação de replicação usando o pagamento para jogadores hiper-racionais $\eqref{defEqRepHR}$ e demonstramos a existência e unicidade da solução em $\Delta$, a invariância de $\Delta_M$, que hiperequilíbrios e estratégias puras são pontos fixos da equação e que a equação de replicação clássica é um caso especial da equação para jogadores hiper-racionais. Usando a equação $\eqref{defEqRepHR}$ para simular o jogo dos colegas de quarto nós observamos como as hiperpreferências mudam o resultado do jogo, levando um jogador a preferir um pagamento efetivo menor somente para reduzir o pagamento de um adversário. Em seguida fizemos uma comparação entre o modelo introduzido por Madeo e Mocenni \cite{madeo2015} e a equação de replicação hiper-racional, onde vimos que a hiper-racionalidade aumenta a importância das ligações do grafo para as quais o jogador possui alguma preferência, geralmente causando mudanças no estado de equilíbrio do jogo.

A análise do jogo estático termina com a afirmação de que o jogador racional seria capaz de, eventualmente, identificar e explorar o altruísmo de um jogador hiper-racional. Uma oportunidade para um trabalho futuro que temos é a análise desses jogos iterados e, até mesmo, de como as estratégias do torneio de Axelrod se comportam com a inserção de jogadores hiper-racionais. Podem ser feitas competições mistas, com jogadores racionais e hiper-racionais, onde podemos dividir os vencedores em efetivos e virtuais, para o pagamento efetivo e para o pagamento virtual respectivamente, e avaliar se jogadores hiper-racionais se o altruísmo é capaz de aumentar os pagamentos individuais e globais do jogo.

Em 2020 foi publicado em \cite{tilman2020evolutionary} um modelo para jogos evolucionários com feedbacks ambientais, que é uma dinâmica comum em sistemas socioecológicos, evolucionário-ecológicos e psicológico-econômicos. A inserção de jogadores hiper-racionais nesse contexto pode mudar os equilíbrios e ciclos do jogo, provendo uma nova visão sobre o problema.

Além disso, neste trabalho não analisamos a estabilidade das soluções encontradas para jogadores hiper-racionais, nem o efeito da rede nessas soluções. Essa é mais uma oportunidade para trabalhos futuros, junto da análise dos efeitos da hiper-racionalidade em outros jogos e, até mesmo, na aplicabilidade deste modelo em problemas reais nas diversas áreas citadas nas quais a teoria dos jogos possui aplicações.
